\documentclass{article}
\usepackage[utf8]{inputenc}
\usepackage{macros}
\usepackage{mathtools}
\usepackage{amssymb}
\usepackage{amsmath}
\usepackage{color}

\begin{document}

\section{TATA41 - Föreläsning 2}
\section{Kontinuitet}
Ordet används i vardagligt tal, utan hopp.
Rita en graf utan att lyfta på pennan.

\subsection{Definition}
Låt $a\in D_f$. Funktionen $f$ sägs vara kontinuerlig i punkten a
ifall $f(x)\to f(a)$ då $x\to a$
(eller, som ett specialfall, om a är en så kallad isolerad punkt i $D_f$.)

\subsubsection{Isolerad punkt}
Vad är en isolerad punkt? Man skulle kunna säga att det är en punkt i en mängd där en punkt inte har några "grannar".
\subsubsection{Ekvivalent}
Låt $a\in D_f$. Funktionen $f$ sägs vara kontinuerlig i punkten a
ifall det för varje $\epsilon>0$ finns ett $\delta>0$ sådant att $\abs{f(x)-f(a)}<\epsilon$ om $x\in D_f$ och $\abs{x-a}<\delta$.

Annars sägs $f$ vara diskontinuerlig i punkten a.

\subsubsection{Anmärkning}
Om $x\notin D_f$ så är det inte meningsfullt att tala om kontinuitet eller diskontinuitet för $f$ i den punkten.

\subsection{Exempel}
Låt
$$
f(x)=
\begin{cases}
  5, x\neq 3,\\
  6, x=3,
\end{cases}
$$

Vilkoret för kontinuitet i punkten $x=3$ är att $f(x)\to f(3)$ då $x\to 3$.
Men detta är inte uppfyllt för vi vet från förra gången att $f(x)\to 5$ då $x\to 3$.
Alltså är $f$ diskontinuerlig i punkten $x=3$.

\subsection{Extra teori}
Låt
$$
f(x)=
\begin{cases}
  x, x\in \mathbb Q \\
  0
\end{cases}
$$
Kontinuerlig i $x=0$ (enbart).

\subsection{Definition av kontinuerlig}
En funktion kallas kontinuerlig (rätt och slätt) om den är kontinuerlig i varje punkt
i sin definionsmängd.

\subsection{Extra teori}
Låt
$$
f(x)=
\begin{cases}
  x\times \sin{\fr 1x}, x\neq 0\\
  0, x=0
\end{cases}
$$

\subsection{Sats (Mellanliggande värde)}
En kontinuerlig funktion på ett intervall i $\R$ antar alla mellanliggande värden,
dvs om $f$ är  kontinuerlig på $\left[ a, b \right]$ och $f(a) < C < f(b)$
eller $f(a) > C > f(b)$ så finns det (minst) ett reellt tal $\xi \in \left[ a,b \right]$
sådant att $f(\xi) = C$.

\subsection{Bevis}
Lokalisera $\xi$ genom intervallhalvering (likt binärsökning). Fullständigheten hos de reella talsystemet
behövs för att visa att processen verkligen konvergerar mot något tal $\xi\in\R$. Se Appendix A i boken.

\subsection{Sats}
Om $f$ och g är kontinuerliga så är följande funktioner kontinuerliga (i de punkter där de är definerade):\\
$f+g, f-g, fg, \fr fg, f\circ g$

\subsubsection{Definition av $f\circ g$}
$(f\circ g)(x) = f(g(x))$

\subsection{Sats}
De elementära funktionerna är kontinuerliga.

\subsection{Bevis}
Se boken

\subsection{Anmärkning}
Därmed har vi motiverat "$\sqrt{1+\fr 1x}\to\sqrt{1+0}$ då $x\to\infty$" från förra gången.

\subsection{Exempel (Styckvis definerad funktion)}
$$
f(x)=
\begin{cases}
  x^2+2, x/le 0,\\
  e^x, x>0,
\end{cases}
$$
är kontinuerlig för $x<0$ och för $x>0$ enligt föregående sats (elementära funktioner).
Funktionen är diskontinuerlig i punkten $x=0$ (eftersom den saknar gränsvärde där).

\subsection{Exempel}
Funktionen $f:\R/\ \{0\} \to \R$ som ges av $f(x)=\f 1x$ är kontinuerlig.

\subsection{Exempel}
Visa att polynomet $p(x) = x^3 + x - 3$ har exakt ett reellt nollställe, och att det ligger i öppna intervallet $\left] 1,2 \right[$.
Lösning:
$$
\begin{cases}
  p(1)=-1, \\
  p(2)=7 \\
\end{cases}
$$
Eftersom $-1<0<7$ och alla polynom är kontinuerliga, så säger satsen om mellanliggande värde att det finns
minst ett tal $\xi\in\left]1,2\right[$ sådant att $p(\xi)=0$.
Vidare är p strängt växande (ty summa av strängt växande funktioner $ g(x) = x^3$ och $h(x) = x-3$),
vilket visar att $p(x)<0$ om $x<\xi$, och $p(x)>0$ om $x>\xi$. Alltså är $\xi$ det enda nollstället.

\subsection{Sats (Största och minsta värde)}
En kontinuerlig funktion på ett kompakt (dvs slutet och begränsat) intervall
har ett största och ett minsta värde, dvs om $f$ är kontinuerlig på $\left[ a,b\right]$
så finns det punkter $c\in\left[ a,b \right]$ och $b\in\left[ a,b \right]$ sådana att
$f(c)\le f(x)\le f(d), \all x \in \left[ a,b \right]$.

\subsection{Bevis}
Se boken (appendix A).

\subsection{Exempel}
På det kompakta intervallet [-3,2] har den kontinuerliga funktion $f(x) = x^2$ största värdet $f(-3)=9$ och minsta värdet $f(0)=0$.
Alla förutsättningarna är nödvändiga för att det garanterat ska finnas största och minsta värde, vilket följande tre exempel visar.

\subsubsection{Exempel (Ej slutet intervall)}
Funktionen $f(x)=x^3$ saknar största och minsta värde på intervallet $]0,2[$.
(Men infimum är 0 och supremum är 4).

\subsubsection{Exempel (Obegänsat intervall)}
Funktionen $f(x)=\f 1x$ saknar minsta värde på intervallet $[1,\infty]$.
("Råkar" dock har strörsta värde $f(x)=1$).

\subsubsection{Exempel (Diskontinuerlig funktionen)}
"Bråkdelsfunktionen" $f(x)=x-\floor x$ saknar största värde intervallet $[\f 12, \fr 32]$

\end{document}
