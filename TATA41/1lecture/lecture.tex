\documentclass{article}
\usepackage[utf8]{inputenc}
\usepackage{macros}
\usepackage{amssymb}
\usepackage{amsmath}
\usepackage{color}

\begin{document}

\section{TATA41 - Föreläsning 1}
Föreläsaren heter Hans gunnmark. Det är samma bok som grunken.\\
Envariabels analys skapades 1700-talet, av Newton och Libnitz.\\

\section{Gränsvärden}
Slå in ett tal i miniräknaren, tryck på roten ur knappen flera gånger så kommer talet närmare och närmare 1.
Det kommer aldrig riktigt till 1, det går mot 1.

\subsection{Exempel}
$\fr 1x$ går mot 0 då $x$ går mot $\infty$\\
$\fr{x^3-1}{x-1}$ går mot 3 då $x$ går mot 1

\subsection{Definition av gränsvärde}
Låt $f$ vara en funktion vars definitionsmängd $D_f$ innehåller godtyckligt stora tal
(dvs $D_f$ är inte en uppåt begränsad mängd).

Påståendet "$f(x) \to A, x \to \infty$" betyder följande:\\
För varje $\epsilon>0$ (oavsett hur litet) finns ett tal $\omega$ (som får bero på $\epsilon$)
sådant att $A-\epsilon < f(x) < A+\epsilon, x>\omega$ och $x\in D_f$

\subsection{Exempel}
Låt $f(x)=\fr 1x$ och undersök om ovanstående vilkor är uppfyllt ifall A=0 resp. A=5.\\
$$ D_f = \R \ \{0\} = \{x\in\R:x\neq 0\} $$ är ej uppåt begränsad. Ok.\\

\subsubsection{Fallet A=0}
Låt $\epsilon > 0$ vara godtyckligt, och betrakta olikheten $A-\epsilon<f(x)<A+\epsilon$,\\
dvs $-\epsilon<\fr 1x<\epsilon$.\\
Denna olikhet är sann för alla $x > \fr 1\epsilon$ (och för alla $x<\fr{-1}{\epsilon}$, men det är irrelevant här).
Vilkoret är därmed uppfyllt (med $\omega=\fr 1\epsilon$), vilket betyder att påståendet "$\fr 1x \to 0, x\to\infty$" är sant.

\subsubsection{Fallet A=5}
Betrakta olikhet $A-\epsilon < f(x) < A+\epsilon$ för (t.ex.) $\epsilon=1$
$4 < \fr 1x < 6 \eq \fr 16 < x < \fr 14$
För detta val av $\epsilon$ finns det inget $\omega$ sådant att olikheten är uppfylld för alla $x>\omega$ Vilkoret är därmed ej uppfyllt, så
"$\fr 1x \to 5, x\to\infty$" är falskt.

\subsection{Sats}

Om $f(x)\to A_1$, och $f(x)\to A_2, x\to\infty$, så är $A_1 = A_2$

\subsection{Bevis}
Vi antar $A_1\neq A_2$ och härleder en motsägelse.\\
Låt $\epsilon>0$ vara så litet att intervallen $\left[ A_1 - \epsilon , A_1 + \epsilon \right]$ och $\left[ A_2 - \epsilon , A_2 + \epsilon \right]$ inte överlappar.\\
Förutsättningen ($f\to A_1$ och $A_2$) innebär (enl. def ovan) att det finns $\omega_1$ och $\omega_2$.\\
Så att:
$$
\begin{cases}
  A_1-\epsilon < f(x) < A_1+\epsilon, \all x>\omega_1; (x\in D_f)\\
  A_2-\epsilon < f(x) < A_2+\epsilon, \all x>\omega_2; (x\in D_f)\\
\end{cases}
$$
Men båda olikheterna kan ej vara sanna samtidigt, eftersom intervallen inte överlappar.\\
Påståendet "$f(x)\to A, x\to\infty$" kan alltså vara sant för högst ett tal A, och om ett sådant tal finns så kalla sdet för gränsvärdet (limes) av $f(x), x\to\infty$
Skrivsätt:
$$\lip{x}f(x) = A$$

Obs! Blanda inte ihop de två skrivsätten. Dvs skriv inte skrivsätten t.ex. $\lip{x} A$ eftersom $\lm{x}{\infty} f(x) = A$ (är gränsvärdet).
Det är $f(x)$ som går mot A inte lim.

\subsection{Räkneregler}
Man använder nästan aldrig definitionen direkt för att räkna ut gränsvärden. Istället används räknelagar.

\subsection{Sats}
Om $f(x)\to A$ och $g(x)\to B, x\to\infty$, så gäller även följande då $x\to\infty$.\\
* $f(x)+g(x)\to A+B$\\
* $f(x)\times g(x)\to A\times B$ (och $cf(x) \to cA$, c=konstant)\\
* $\fr{f(x)}{g(x)}\to \fr AB$ om $B\neq0$\\
Under förutsättning att vänsterledens def.mängder innehåller godt, stora tal.

\subsection{Bevis för räknelagen för plus}
Låt $\epsilon>0$ Sätt $\phi=\fr \epsilon2$ Gränsvärdes definitionen ger att det finns $\omega_1$ och $\omega_2$ så att
$$
\begin{cases}
  A-\phi < f(x) < A+\phi, x>\omega_1, x\in D_f,\\
  B-\phi < g(x) < B+\phi, x>\omega_2,  x\in D_g
\end{cases}
$$

Ledvis addition ger att
$$ (A-\phi)+(B-\phi) < f(x) + g(x) < (A+\phi)+(B+\phi) $$ om $x>\omega_1$ och $x>\omega_2$ och $x\in D_f$ och $x\in D_g$
dvs (eftersom $\phi+\phi + \fr \epsilon2 + \fr \epsilon2 = \epsilon$)
$$ (A+B)-\epsilon< f(x) + g(x) < (A+B)+\epsilon) $$ om $x>\omega = \max(\omega_1, \omega_2)$ och $x\in D_{f+g}$

\subsection{Exempel}
Vi har redan visat att $\fr 1x \to 0$ då $x\to\infty$.
Räknelagen för gånger ger då
$$ \fr 1{x^2}=\fr 1x \times \fr 1x \to 0\times 0 = 0, x\to\infty $$
$$ \fr 1{x^3}=\fr 1{x^2} \times \fr 1x \to 0\times 0 = 0, x\to\infty $$
osv.\\

Även t.ex. $\fr 1{x^{\fr 12}} = \fr 1{\sqrt x} \to x\to \infty$.
Räknelagarna ger även t.ex.
$$ \fr{2x^2-3x+1}{5x^2+4} = \fr{x^2(2-\fr 3x + \fr 1{x^2})}{x^2(5+\fr 4{x^2})} \to \fr{2-0+0}{5+0} = \fr 25, x\to\infty $$
och\\

$ \sqrt{x^2+x} - x = \fr{(\sqrt{x^2} + x)(\sqrt{x^2} - x)}{\sqrt{x^2} + x} = \fr{(x^2 + x) - x^2}{\sqrt{x^2+x} + x} = \fr{x}{ \sqrt{x^2} \sqrt{1+\fr 1x} + x } = \fr{x}{x(\sqrt{1+\fr 1x} + 1)} =
\fr{1}{\sqrt{1+\fr 1x} + 1} \to \fr{1}{\sqrt{1+0} + 1} = \fr 12, x\to\infty $\\

Vi kan anta att $x>0$ när vi ska låta $x\to\infty$. Då är $\sqrt{x^2} = \abs{x} = x$\\
I sista steget används att $\sqrt x$ är en kontinuerlig funtion. Mer om det nästa gång.\\

Se boken för det def. av gr.v då $x\to -\infty, x\to a$ (ett reellt tal),
då $x\to a^+$ resp $a^-$ (höger- resp vänster-gränsvärde), och av s.k. oegentligt gr.v.
$f(x)\to\infty$ eller $-\infty$.

\subsection{Exempel}
$$ \lm{x}{2} \fr{x^2+x-6}{x^2-4} = \fr 54 $$ty
$$ \fr{x^2+x-6}{x^2-4} = \fr{(x-2)(x+3)}{(x-2)(x+2)} = \fr{x+3}{x+2} \to \fr{2+3}{2+2} = \fr 54, x\to 2$$

\subsection{Exempel}

% TODO: graf av y=1\x
Låt $f(x)=\fr 1x$ (för $x\neq 0$)\\
* $f(x)\to \infty$ då $x\to 0^+$\\
* $f(x)\to -\infty$ då $x\to 0^-$\\
* $f(x)$ har inget gr.v. (inte ens oengentligt) då $x\to 0$

\subsection{Exempel}
Låt:
% TODO: graf av y=f(x), 6 och 5 markerat på y, 3 på x
$$
f(x) =
\begin{cases}
  5, x \neq 3,\\
  6, x = 3,\\
\end{cases}
$$

Då är $\lm x3 f(x) =5$ (Att f(3) = 6 påverkar per def. inte gränsvärdet)

\end{document}
