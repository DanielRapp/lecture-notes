\documentclass{article}
\usepackage[normalem]{ulem}
\usepackage[utf8]{inputenc}
\usepackage{graphicx}
\usepackage{mathtools}
\usepackage{amssymb}
\usepackage{amsmath}
\usepackage{macros}
\usepackage{color}
\newcommand{\Or}{{\mathcal{O}}}
\newcommand{\RR}{{\mathcal{R}}}
\newcommand{\N}{{\mathcal{N}}}
\newcommand{\intf}{\int^{\infty}_{-\infty}}
\newcommand{\intb}{\int^{1}_{0}}

\begin{document}
%Om C är konstig (t.ex. mängden av alla irrationella tal), gäller
%$$ P((X,Y)\in C) = {\int\int}_{(x,y)\in C} f(x,y) dxdy $$
%fortfarande?


$$
X_1\thicksim N(\mu_1, \sigma_1^2), \dots,  X_n\thicksim N(\mu_n, \sigma_n^2) \im
$$
$$
X_1+\dots+X_n \thicksim N(\mu_1+\dots+\mu_n, \sigma_1^2+\dots+\sigma_n^2)
$$


$$
X\thicksim Bin(n, p), Y\thicksim Bin(m, p) \im X+Y \thicksim Bin(n+m, p)
$$
$$
X\thicksim Po(\lambda_1), Y\thicksim Po(\lambda_2) \im X+Y \thicksim Po(\lambda_1 + \lambda_2)
$$

We can approximate the binomial distribution with the normal distribution.
$$ X\thicksim Bin(26, 0.4) $$
$$ E[X] = 26*0.4 = 10.4,\ Var(X) = 26*0.4*0.6 = 6.24 $$
$$ \hat{X}\thicksim \N(10.4, 6.24) $$

\end{document}

