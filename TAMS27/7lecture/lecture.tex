\documentclass{article}
\usepackage[normalem]{ulem}
\usepackage[utf8]{inputenc}
\usepackage{graphicx}
\usepackage{mathtools}
\usepackage{amssymb}
\usepackage{amsmath}
\usepackage{macros}
\usepackage{color}
\newcommand{\Or}{{\mathcal{O}}}
\newcommand{\RR}{{\mathcal{R}}}
\newcommand{\N}{{\mathcal{N}}}
\newcommand{\intf}{\int^{\infty}_{-\infty}}
\newcommand{\intb}{\int^{1}_{0}}

\begin{document}
\section{Simultana fördelningar}
\subsection{Def}
Låt X, Y vara två diskreta slumpvariabler
\begin{itemize}
  \item $ p(x,y) = P(X=x, Y=y), \forall x,y $\\
    kallas den simultana sannolikhetsfunktionen.
  \item $ p_X(x) = P(X=x) = \sum_{\forall y}p(x,y) $\\
        $ p_Y(y) = P(Y=y) = \sum_{\forall x}p(x,y) $\\
        Kallas de marginella sannolikhetsfunktionerna.
\end{itemize}

\subsection{Def}
I en urna finns

\begin{itemize}
  \item 3 röda
  \item 4 vita
  \item 5 blåa kulor
\end{itemize}

X: antalet röda kulor, Y: antalet vita kulor.\\
Ta 3 styckna kulor.
Vi ska bestämma den simultana och de marginella sannolikhetsfunktionerna.\\
\begin{tabular}{ l | l l l l l l l }
  $i,j$       & 0 & 1 & 2 & 3 & $P(X=i)$ & \\\hline
  $0$         &$ \f{10}{220} $&$ \f{40}{220} $&$ \f{30}{220} $&$ \f{4}{220} $&$ \f{84}{220} $& \\
  $1$         &$ \f{30}{220} $&$ \f{60}{220} $&$ \f{18}{220} $&$ 0 $&$ \f{108}{220} $& \\
  $2$         &$ \f{15}{220} $&$ \f{12}{220}  $&$ 0 $&$ 0 $&$ \f{27}{220} $& \\
  $3$         &$ \f{1}{220} $&$ 0 $&$ 0 $&$ 0 $&$ \f{1}{220} $& \\
  $P(Y=j)$    &$ \f{56}{220} $&$ \f{112}{220} $&$ \f{48}{220} $&$ \f{4}{220} $&  & \\
\end{tabular}

\subsection{Def}
Låt X, Y vara två kontinuerliga slumpvariabler.

\begin{itemize}
  \item Om det finns en icke-negativ funktion $f(x,y)$ sådan att
    $$ P((X,Y) \in C) = {\int \int}_{(x,y)\in C}f(x,y)dxdy $$
    För alla möjliga delmängder $C \subseteq \R^2$, så kallas $f$ den simultana täthetsfunktionen av X och Y.
  \item $f_X(x) = \intf f(x,y)dy$
   $f_Y(y) = \intf f(x,y)dx$
   kallas de marginella täthetsfunktionerna.
\end{itemize}

\subsection{Anmärkningar}

\begin{itemize}
  \item I det \underline{diskreta} fallet: $p_X(x), p_Y(y)$ sannolikhetsfunktioner av de (enskilda) slumpvariabler X respektive Y.
  \item I det \underline{kontinuerliga} fallet: $f_X(x), f_Y(y)$ täthetsfunktioner av de enskilda s.v. X respektive Y.
\end{itemize}
Det betyder att
$$ P(X\in A) = \int_{x\in A} f_X(x)dx = \int_{x\in A} \intf f(x,y) dydx $$

\subsection{Example}
Låt
$$
f(x,y)=
\begin{cases}
  2e^{-x}e^{-2y}, 0<x<\infty, 0<y<\infty\\
  0, annars
\end{cases}
$$

vara den simultana täthetsfunktionen av $(X, Y)$.
Bestäm
\begin{itemize}
  \item $P(X>1, Y<1)$
  \item $P(X<Y)$
  \item $P(X<a)$
\end{itemize}

\subsubsection{Lösning (1)}
$$P(X>1, Y<1) =
\int^{1}_{y=0} \int^{\infty}_{x=1} 2e^{-x}e^{-2y}dxdy  = \int^1_{y=0} 2e^{-2y} dy \int^{\infty}_{x=1} e^{-x}dx   = \cdots =$$
$$\pa{1-e^{-2}}e^{-1} = e^{-1}-e^{-3}$$

\newcommand{\intyy}{\int^{\infty}_{y=0}}
\newcommand{\intxx}{\int^{y}_{x=0}}
\newcommand{\intzf}{\int^{\infty}_{0}}
\subsubsection{Lösning (2)}
$$ P(X<Y) = {\int\int}_{(x,y):x<y} 2e^{-x}e^{-2y} dxdy  = \intyy\intxx 2e^{-x}e^{-2y} dxdy =$$
$$ = \intyy 2e^{-2y}dy \intxx e^{-x} dx = \intzf 2e^{-2y}\pa{1-e^{-y}} dy = \cdots = \f 13$$

\newcommand{\intyf}{\int^{\infty}_{-\infty}}
\newcommand{\intyz}{\int^{\infty}_{-\infty}}
\subsubsection{Lösning (3)}
Beräkna först marginella täthetsfunktionen för X.
$$ f_X(x) = \intyf f(x,y)dy = \intyz 2e^{-2y}e^{-x}dy = e^{-x}, x\leq 0 , X\thicksim Exp(1)$$
$$ f_X(x) = 0 , x<0$$
$$ P(X<a) = F_X(a) = 1-e^{-a} $$

\section{Oberoende slumpvariabler}
\subsection{Motivation}

Oberoende händelser E,F
$$ P(E\cap F) = P(E)P(F) $$

\subsection{Definition}
Låt X, Y vara två slumpvariabler. X och Y kallas oberoende om
$$ P(X\in A, Y\in B) = P(X\in A)P(Y\in B), \forall A,b\subseteq \R $$

\subsection{Sats}
\begin{itemize}
  \item \underline{diskret fall} Två diskreta s.v. X och Y är oberoende om och endast om
    $$ p(x,y)=p_X(x)p_Y(y), \forall x,y $$
  \item \underline{kontinuerligt fall} Två s.v. X och Y är oberoende om och endast om
    $$ f(x,y)=f_X(x)f_Y(y), \forall x,y\in\R $$
\end{itemize}

\subsection{Anmärkning}
Diskreta fallet kan skrivas så här
$$ P(X=x, Y=y) = P(X=x)P(Y=y), \forall x,y$$

\subsection{Example}
I en urna finns

\begin{itemize}
  \item 3 röda
  \item 4 vita
  \item 5 blåa kulor
\end{itemize}

Ta ur 3. X: \# röda kulor. Y: \# vita kulor. Är X och Y oberoende?\\
\begin{tabular}{ l | l l l l l l l }
  $i,j$       & 0 & 1 & 2 & 3 & $P_X(i)$ & \\\hline
  $0$         &$ \f{10}{220} $&$ \f{40}{220} $&$ \f{30}{220} $&$ \f{4}{220} $&$ \f{84}{220} $& \\
  $1$         &$ \f{30}{220} $&$ \f{60}{220} $&$ \f{18}{220} $&$ 0 $&$ \f{108}{220} $& \\
  $2$         &$ \f{15}{220} $&$ \f{12}{220}  $&$ 0 $&$ 0 $&$ \f{27}{220} $& \\
  $3$         &$ \f{1}{220} $&$ 0 $&$ 0 $&$ 0 $&$ \f{1}{220} $& \\
  $p_Y(j)$    &$ \f{56}{220} $&$ \f{112}{220} $&$ \f{48}{220} $&$ \f{4}{220} $&  & \\
\end{tabular}\\
Inte oberoende eftersom (t.ex.)
$$ p(3,3) = 0 \neq p_X(3)p_Y(3) \neq 0 $$

\subsection{Anmärkning}
Om man vill visa att slumpvariabler X, Y \underline{inte} är oberoende räcker \underline{ett enda motexempel}.

\subsection{Example}
$$
f(x,y)=
\begin{cases}
  2e^{-x}e^{-2y}, 0<x<\infty, 0<y<\infty\\
  0, annars
\end{cases}
$$
Är X och Y oberoende?

\subsubsection{Lösning}
Har redan beräknat
$$
f_X(x)=
\begin{cases}
  e^{-x} x\ge 0
  0, annars
\end{cases}
$$

\newcommand{\intzxf}{\int^{\infty}_{x=-\infty}}
\newcommand{\intxxf}{\int^{\infty}_{x=0}}
$$
y\ge 0; f_Y(y)=\intzxf f(x,y) dx = \intxxz 2e^{-x}e^{-2y} dx = 2e^{-2y} \intxxf e^{-x} dx = 2e^{-2y} , Y\thicksim Exp(2)
$$

$$
y\leq 0; f_Y(y)=0
$$

Vi konstaterar
$$
f_X(x)f_Y(y)=
\begin{cases}
  e^{-x}2e^{-2y}  0<x<\infty, 0<y<\infty\\
  0, annars
\end{cases}
$$
Alltså $f_X(x)f_Y(y) = f(x,y)$, X och Y är oberoende.

\subsection{Example}
Man och kvinna vill träffas på ett kafe. Antag att båda kommer oberoende av varandra (likformigt fördelad mellan kl 12 och kl 13).
Bestäm sannolikheten att den som kommer först behöver vänta ländre än 10 min.

\subsection{Lösning}
X: ankomsttid av mannen i minuter efter kl 12.\\
Y: ankomsttid av kvinnan i minuter efter kl 12.\\
$$ X\thicksim U[0,60], Y\thicksim [0,60], X, Y oberoende \im $$
$$ f_X(x) = 

\begin{cases}
  \f 1{60} 0\leq x \leq 60\\
  \f 0, annars
\end{cases}$$

$$ f_X(x) = 

\begin{cases}
  \f 1{60} 0\leq y \leq 60\\
  \f 0, annars
\end{cases}$$

X,Y oberoende:
$$ f(x,y) = 

\begin{cases}
  \f 1{60^2} 0\leq x \leq 60, 0\leq y \leq 60\\
  \f 0, annars
\end{cases}$$

Vi söker
$$ P(\bra{X+10\le Y} \cup \bra{Y+10\le X}) = 2P(X+10\le Y) $$
$$ 2{\int\int}_{x+10<y} \f 1{60^2}dxdy = \f{25}{36} \pa{>\f23} $$

\end{document}

