\documentclass{article}
\usepackage[normalem]{ulem}
\usepackage[utf8]{inputenc}
\usepackage{graphicx}
\usepackage{mathtools}
\usepackage{amssymb}
\usepackage{amsmath}
\usepackage{macros}
\usepackage{color}
\newcommand{\Or}{{\mathcal{O}}}
\newcommand{\RR}{{\mathcal{R}}}
\newcommand{\N}{{\mathcal{N}}}
\newcommand{\intf}{\int^{\infty}_{-\infty}}
\newcommand{\intb}{\int^{1}_{0}}

\begin{document}
\section{Kontinuerliga stokastiska variabler}
Discrete:
$$ X:S\rightarrow C\subset \RR $$
A countable set of real numbers.\\
Continous:
$$ X:S\rightarrow I\subseteq \RR $$
I is an uncountable set = with the cardinality of the real numbers.


\subsection{Definition}
\begin{itemize}
    \item X is called a continous random variable if it can take all values from a real interval or union of F (finite) intervals.\\
      Example: Lifespan of a lightbulb. Weight of a raindrop. Circumference of a randomly chosen tree.
    \item A function $f_x:\RR \rightarrow [0,\infty)$ which satisfies
        $$ P(a\le X\le b) = \int^b_a f_x(x)dx $$
        is called "density" function.
      \item $F_X(x) = P(X\le x) = \int^x_{-\infty}f_X(y)dy$
        is called the "comulative" distribution function.
        $$P(a\le X\le b) = P(X\le b) - P(X\le a) = F_X(b)- F_X(a)$$
\end{itemize}

\subsection{Properties}
\begin{itemize}
    \item $ f_X(x) \le 0 $
    \item $ \int^\infty_{-\infty} f_X(x)dx = 1 $
    \item $P(a<X\le b) = P(a<X<b) = P(a\le X\le b) = P(a\le X<b)  $
\end{itemize}

\subsection{Remark}
Since $f_X(x)$ is non-negative $\int f_X$ is an area

\subsection{Example}
Assume that X is a continous random variable with density function
$$ f(x) =
  \begin{cases}
    c(4x-2x^2), 0<x<2\\
    0, otherwise
  \end{cases}
$$
Find the constant c and $P(X>1)$
$$ \int^\infty_{-\infty} f(x) = 1 \rightarrow 1 = \int^\infty_{-\infty} c(4x-2x^2)dx-\int^2_0c(4x-2x^2)dx =$$
$$= {\left[ c(2x^2 - \f 23 x^3) \right]}^2_0 = c(8-\f {16}3) = 1 \im c = \f 38  $$
Now our density is
$$ f(x) =
  \begin{cases}
    \f 38(4x-2x^2), 0<x<2\\
    0, otherwise
  \end{cases}
$$
$$ P(X>1) = \int^2_1 f(x)dx =\f 38 \int^2_1 \f 38(4x+2x^2) =\dots =\f12  $$

\section{Expectation and variance}
\subsection{Definitions}
Let X be a continous random variable.
\begin{itemize}
    \item $ E[X] = \int^{\infty}_{-\infty} xf_X(x)dx $ \\is called expectation (or expected value) of X.
    \item $ Var[X] = \int^{\infty}_{-\infty} (x-E[X])^2f_X(x)dx $ \\is called variance of X.
    \item $ D[X] = \sqrt{Var(X)} $ \\is called standard deviation
\end{itemize}

\subsection{Remark}
$$ Var(X) = E[X^2] - E[X]^2 = \int^\infty_{-\infty} x^2 f_x(x) dx - {\left( \int^\infty_{-\infty} xf_x(x)dx \right)}  $$

\subsection{Theorem}
The first and second properties are the same as in the descrete case.
\begin{itemize}
  \item $ E[a] = a, Var(a) = 0, \forall a \in \R $
  \item $ E[aX+b] = aE[X]+b$
  \item $Var(aX+b) = a^2Var(X)$
  \item $E[g(x)] = \int^\infty_{\infty} g(x)f_X(x)dx$
\end{itemize}

\subsection{Example}
$$ f_X(x) =
  \begin{cases}
    1, 0\le x\le 2\\
    0, otherwise
  \end{cases}
$$
$$ g(x) = e^x. E[e^x] = ?, Var(e^x) = ? $$
\begin{itemize}
  \item $ E[e^X] = \intf (e^x)(f_X(x))dx = \int^1_0 e^x * 1 dx = e - 1 $
  \item $ Var(X) = E[X^2] - E[X]^2 $\\
    $ E[X^2] = E[(e^x)^2] = E[e^{2x}] = \intb e^{2x} * 1 dx  = {\left[ \f 12 e^{2x}\right]}^1_0 = \f 12(e^2-1)$\\
    $ Var = \f 12 (e^2-1) - {(e-1)}^2 $
\end{itemize}

\section{Common distributions}
\subsection{Uniform distribution}
Uniform distribution on $[a,b]$ (the symbol $U([a,b])$)
\subsubsection{Definition}
$$ f_X(x) =
  \begin{cases}
    \f 1{b-a}, a\le x\le b\\
    0, otherwise
  \end{cases}
$$

\subsubsection{Theorem}
\begin{itemize}
  \item  $ f_X(x) =
  \begin{cases}
    0, -\infty \le x < a
    \f {x-a}{b-a}, a\le x\le b\\
    1, x\ge b
  \end{cases}$
 \item $ E[X] = \f{a+b}2, Var(X) = \f {{(b-a)}^2}{12} $
\end{itemize}

{\color{red} Proof not included}

\subsection{Exponential distribution}
$$ X\thicksim Exp(\lambda) $$
Given a $0 < \lambda \in \R$
$$ f_X(x) = \begin{cases}
  \lambda e^{-\lambda x},\ x\ge 0\\
  0,\ otherwise
\end{cases} $$

\subsubsection{Theorem}
\begin{itemize}
    \item $ F_X(x) = \begin{cases} 
      0,\ -\infty\le x\le 0\\
      1-e^{-\lambda x},\ x\ge 0
    \end{cases} $
  \item $ E[X] = \f 1{\lambda}; Var(X) = \f 1{\lambda^2}$
\end{itemize}

\subsubsection{Theorem (Memoryless property)}
Let $X\thicksim Exp(\lambda)$. We have that $P(X>t+x | X>t) = P(X>x)$
\subsubsection{Proof}
$$P(X>t+x | X>t) = \f{P(X>t+x ; X>t)}{P(X>t)} = \f{P(X>t+x)}{P(X>t)} = \f{1-F_X(t+x)}{1-F_X(t)} =$$
$$\f{e^{-\lambda (t+x)}}{e^{-\lambda t}} = e^{-\lambda x} = 1-F_X(x) = P(X>x) \qed$$

\subsection{Standard normal distribution}
$$ X\thicksim \N(0, 1) $$
The random variable has symbol Z, the density funciton is defined as 
$$ f_X(x) = \f 1{\sqrt{2\pi}}e^{-\f{x^2}2}, x\in \R = (-\infty, \infty) $$
$F_X(x)$ doesn't have an explicit formula.\\
The symbol $F_X$ is $\Phi(x) = P(X\le x)$

\subsubsection{Theorem}
Let $ Z\thicksim \N(0, 1) $ then
$$ E(Z) = 0, Var(Z)=1 $$
Proof comes later in the course.

\subsubsection{Example}
$$ Z\thicksim \N(0, 1) $$
\begin{itemize}
    \item $ P(Z<2.11) = \Phi(2.11) = 0.9826,\ from\ table $
      \item $ P(Z\le -2.11) = \Phi(-2.11) = 1-\Phi(2.11) = 1 - 0.9826 $
        \item $ P(-2.21\le Z\le 2.11) = P(Z\le 2.11) - P(Z\le -2.21) = 0.9826  - 0.0163 = 0.969 $
\end{itemize}

\section{Standard deveation}
$$ X\thicksim f_X(x) = \f 1 {(2\pi)\sigma}e^{ -\f{x-\mu}{2\sigma^2} } $$
$$ X\thicksim \N(\mu, \sigma^2)$$

\subsection{Theorem}
$$ E[X] = \mu, Var(X) = \sigma^2 $$
How do we compute $F_X(x)$?\\
Standardization: we transform X in Z
$$ Z = \f{x-\mu}{\sigma} \thicksim \N(0,1) $$

\end{document}
