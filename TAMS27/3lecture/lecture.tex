\documentclass{article}
\usepackage[normalem]{ulem}
\usepackage[utf8]{inputenc}
\usepackage{graphicx}
\usepackage{mathtools}
\usepackage{amssymb}
\usepackage{amsmath}
\usepackage{macros}
\usepackage{color}
\newcommand{\Or}{{\mathcal{O}}}

\begin{document}
\section{Betingad sannolikhet}
\subsection{Definition}
Låt E och F vara två händelser.
Antag $P(F)>0$. Sannolikheten för E \underline{betingat} av F betecknas med $P(E|F)$ och defineras som
$$ P(E|F) = \f{P(EF)}{P(F)} $$
\subsection{Exempel}
Kasta en tärning E = \{ fått en etta \}, F = \{ fått ett ojämt antal ögon \}
$$ P(E) = \f16, P(F) = \f12, P(EF) = \f16 $$
$$ P(E|F) = \f{P(EF)}{P(F)} = \f13 \qed $$
\subsection{Egenskaper}
\subsubsection{Sats (a) (Lagen om total sannolikhet)}
Låt $F_1, F_2, \dots, F_n$ disjunkta händelser med $P(F_i)>0, i=i,\dots,n,$ som uppfyller hela utfallsrummet (dvs $\cup^n_{i=1}F_i = S$). För varje händelse E gäller
$$ P(E) = \sum_{i=1}^{n} P(E|F_i)P(F_i) $$

\subsubsection{Sats (b) (Bayes' formel)}
Under samma villkor som (a)
$$ P(F_i|E) = \f{P(E|F_i)P(F_i)}{P(E)} $$

\subsubsection{Bevis (a)}
HL
$$ \sum_{i=1}^{n} P(E|F_i)*P(F_i) = \sum_{i=1}^{n} \f{P(EF_i)}{P(F_i)}*P(F_i) = \sum_{i=1}^{n} P(EF_i) = P(\cup^n_{i=n}EF_i) = $$
$$ P(E(P(\cup^n_{i=n}F_i))) = P(ES) = P(E) \qed$$

\subsubsection{Bevis (b)}
HL
$$ \f{P(E|F_i)P(F_i)}{P(E)} = \f{\f{P(EF_i)}{P(F_i)}*P(F_i)}{P(E)} = \f{P(EF_i)}{P(E)} = P(F_i|E) \qed $$

\subsection{Exempel}
I en fabrik tillverkas 25\% av enheterna vid maskin 1, 35\% vid maskin 2 och 40\% vid maskin 3. Av produktionen är respektive 5\%, 4\% och 2\% defekt.\\

Total sannolikhet
$$ P(E) = P(E|F_1)P(F_1) + P(E|F_2)P(F_2) + P(E|F_3)P(F_3)$$
\subsubsection{Fråga (a)}
Hur stor är sannolikheten att en slumpmässigt vald enhet är defekt?

\subsubsection{Fråga (b)}
Antag att en kund påträffar en felaktig enhet. Hur stor är sannolikheten att den tillverkades av maskin 1?

$$ \f{P(E|F_i)*P(F_i)}{P(E)} = P(F_i|E) $$

\subsubsection{Lösning (a)}
Rimligt ($i=1,2,3$)\\
$F_i$ = \{ enhet var tillverkats vid maskin i \}\\
$E_i$ = \{ enhet är felaktig \}\\
Från text:
$$ P(F_1)=0.25, P(F_2)=0.35, P(F_3)=0.4 $$
Efter "omformulering":
$$ P(E|F_1) = 0.05, P(E|F_2) = 0.04, P(E|F_3) = 0.02 $$
Total sannolikhet ger
$$ P(E) = P(E|F_1)P(F_1) + P(E|F_2)P(F_2) +P(E|F_3)P(F_3) = $$
$$ = 0.05*0.25+0.04*0.35+0.02*0.4=0.0345 \qed $$

\subsubsection{Lösning (b)}
"givet att enheten är felaktig"
$$ P(\ \ |E) $$
"hur stor sannolikhet att enheten har tillverkats vid maskin 1?"
$$ P(F_i|E) =$$
$$= \f{P(E|F_i)P(F_i)}{P(E)} = \f{0.05*0.25}{0.0345} = 0.36 \qed $$

\subsection{Oberoende händelser}
Om betingad sannolikhet
$$ P(E|F) = P(E) $$
så påverkar F inte sannolikheten för E.\\
Vi säger att E och F är oberoende om
$$ P(E|F) = P(E) $$
\subsection{Definition}
Två händelser E och F säges vara \underline{oberoende} om
$$ P(E\cap F) = P(E)P(F) $$

\subsection{Exempel}
Två enheter $B_1$ och $B_2$ är parallelkopplade. $B_1$ och $B_2$ fungerar oberoende av varandra (fysikaliskt).
Vi antar att
$$ P(\{ enhet\ B_i\ fungerar \}) = 0.9 $$
Vi säger att systemet fungerar om minst en av $B_1$ och $B_2$ fungerar. Besäm sannolikheten att systemet fungerar.
\subsubsection{Lösning}
$$ A_1=\{ B_1\ fungerar\} $$
$$ A_2=\{ B_2\ fungerar\} $$
Vi ska bestämma
$$ P(A_1\cup A_2) $$
Slut av föreläsning 2:
$$ = P(A_1) + P(A_2) - P(A_1 \cap A_2) = 0.9+0.9+0.81 = 0.99 $$

\section{Stokastiska variabler (= slumpvariabler)}
\subsection{Definition}
En stokastisk variabel X är en funktion från utfallsrummet till den reella linjen $\mathcal{R}$, dvs $X: \Omega \rightarrow \mathcal{R}$.
\subsection{Exempel}
\begin{itemize}
    \item X=pH-värdet i ett vattenprov.\\
    \item Y=antalet studenter som lyckas genomföra en viss laboration inom en utsatt tid.
\end{itemize}
\subsection{Exempel}
Kasta en tärning\\
$E_1 =$ \{ fått en etta \} $\longrightarrow\ 1, X(E_1)=1$\\
$\dots$ \\
$E_6 =$ \{ fått en sexa\} $\longrightarrow\ 6, X(E_6)=6$\\\\
Man säger också $P(X=1)=\f16\dots P(X=6)=\f16$.

\subsection{Definition (a)}
En stokastisk variabel säges vara \underline{diskret} om den kan anta ett ändligt eller uppräkneligt antal olika värden.

\subsection{Definition (b)}
För en diskrev stokastisk variabel defineras \underline{sannolikhetsfunktionen}
$$ p_X(k) = P(X=k), \forall k\in Image(X)$$

\subsection{Exempel}
Gör två kast med ett mynt. Låt
$$ X\ :\ \#(antalet)\ kast\ som\ ger\ krona $$
X kan anta värdena $0, 1, 2$

\subsubsection{Sannolikhetsfunktion}
$$ p_X(0) = P(X=0) = P(\{ kl, kl \}) = \f14 $$
$$ p_X(1) = P(X=1) = P(\{ kl, kr \} \cup \{ kr, kl \}) $$
$$ = P(\{ kl, kr \}) + P(\{ kr, kl \}) = \f 12$$
$$ p_X(2) = P(X=2) = P(\{ kr, kr \}) = \f 14$$

\subsection{Exempel}
Sannolikhetsfunktionen av en viss stokastisk variabel ges av 
$$ P(X=k) = c\f{\lambda^k}{k!},\ k=0,1,2,\dots,\ \lambda>0 $$
Bestäm C så att summan av alla P = 1.
$$ 1 = \sum_{k=0}^{\inf} c\f{\lambda^k}{k!}= c\sum_{k=0}^{\inf} \f{\lambda^k}{k!} = c=e^\lambda \implies c=e^{-\lambda}  $$

\end{document}
