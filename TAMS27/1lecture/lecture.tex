\documentclass{article}
\usepackage[normalem]{ulem}
\usepackage[utf8]{inputenc}
\usepackage{graphicx}
\usepackage{mathtools}
\usepackage{amssymb}
\usepackage{amsmath}
\usepackage{macros}
\usepackage{color}
\newcommand{\Or}{{\mathcal{O}}}

\begin{document}
\section{TAMS27}
\subsection{Kombinatorik}
Hur många möjligheter finns?
\subsection{Definition}
\subsubsection{Permutation}
En permutation är ett arrangemang av ett antal objekt \underline{med} hänsyn till deras ordning.
\subsubsection{Kombination}
En kombination är ett val at ett antal objekt \underline{utan} hänsyn till deras ordning.

\subsection{Ex}
Alla permutationer av objekten A, B, C:\\
$$ ABC, ACB, BAC, BCA, CAB, CBA $$
Alla kombinationer: 
$$ ABC $$

\subsection{Ex}
Vi har 3 objekt A, B, C. Vi ska välja ut 2 st och ordna dem.
Vi får följande möjligheter:
$$ AB, BA, AC, CA, BC, CB $$
Vi ska välja 2 st utan att ordna dem.
$$ AB, BC, AC $$

\subsection{Multiplikationsprincipen}
Betrakta ett experiment som utförs i k steg. Låt för $ i = 1, \cdots, k $ och $ n_i $ vara antalet sätt som steg i kan utföras på.
Det totala antalet sätt som experimentet kan utföras på blir då.
$$ n = n_i * \cdots * n_k $$

\subsection{Beteckning}
$$ n! = 1*2* \cdots *n $$
$$ 0! = 1 $$

\subsection{Sats}
\subsubsection{A}
Antalet sätt att arrangera i ordning (d.v.s. permutera) r olika objekt valda ur en mängd med n stycken är:
$$ \f{n!}{(n-r)!} = n*(n-1)*\cdots*(n-r+1) $$
\subsubsection{B}
Antalet sätt att välja r olika objekt utan återläggning ur en mängd med n stycken utan hänsyn till deras inbörda ordning.
$$ {n \choose r} = \f{n!}{(n-r)!r!} = \f{n*(n-1)*\cdots*(n-r+1)}{r!} $$

\subsection{Beviset}
\subsubsection{A}
Beviset av A delen med multiplikationsprincipen.
\subsubsection{B}
Beviset av B följer ifrån observationenn $ \# permutationer = \# kombinationer * r! $

\subsection{Exempel}
5 bilar anländer till en parkeringsplats med 7 enskilda platser. På hur många sätt kan bilarna placeras?
Svar: Sats A:
$$ \f{7!}{(7-5)!} = 2520 $$

\subsection{Exempel}
Man har 20 motorer av vilka 5 är defekta. Man väljer ut 3 st för kontroll. Vad är antalet sätt att välja ut 3 felfria?
Svar: Sats B ger:
$$ {15 \choose 3} = 455 $$
Vad är antalet sätt att få 2 felfria och 1 defekt?
Svar: Sats B och multiplikationsprincipen ger:
$$ {15 \choose 2}*{5 \choose 1} = 105 * 5 = 525 $$

\subsection{Permutationer med upprepningar}
Man har k grupper som består av resketive $ n_1, n_2, \dots , n_k $ lika objekt. Låt $ n = n_1 + n_2 + \dots + n_k$

\subsection{Sats}
Antalet sätt att arrangera i ordning (d.v.s. permutera) de n st objekten är
$$ \f{n!}{n_1!*n_2!*\dots*n_k!} $$

\subsubsection{Sats}
Hur många permutationer av ordet ABRAKADABRA finns? 5 grupper. A, B, R, K, D.
$$ n_1 = 5\ (\#A) $$
$$ n_2 = 2\ (\#B) $$
$$ n_3 = 2\ (\#R) $$
$$ n_4 = 1\ (\#R) $$
$$ n_5 = 1\ (\#D) $$

Svaret:
$$ \f{11!}{5!*2!*2!*1!*1!} = 83160 $$

\section{Händelse och utfallsrum}
\subsection{Definition A}
Varje möjligt resultat av ett slumpförsök kallas ett utfall eller elementärhändelse.

\subsection{Definition B}
Mängden av alla utfall eller resultat kallar vi utfallsrum och betecknar det med $S$ eller $\Omega$.

\subsection{Definition C}
En händelse A är en mängd av utfall, d.v.s. en delmängd av $\Omega$.

\subsection{Grundläggande operationer med händelser}
{\color{red}Bilder på venn-diagramm ej ritade, se boken}
\begin{itemize}
    \item A uttalas ``A inträffar''.\\
    \item $A\cap B$ eller $AB$ uttalas ``A och B inträffar'' eller ``A snitt B''.\\
    \item $A\cup B$ eller $A + B$ uttalas ``A eller B inträffar'' eller ``A union B''.\\
    \item $A^\complement$ eller $\overline{A}$ uttalas ``komplement till A''.\\
\end{itemize}

\subsection{Räkneregler}
Kommutativa lagar
$$ E\cup F = F\cup E $$
$$ EF=FE $$\\

Associativa lagar
$$ (E\cup F)\cup G = E\cup (F\cup G) $$
$$ (EF)G = E(FG) $$

Distributiva lagar
$$ (E\cup F)G = EG\cup FG $$
$$ EF\cup G = (E\cup G)(F\cup G) $$

De Morgans lagar
$$ (E\cup F)^\complement = E^\complement F^\complement $$
$$ (EF)^\complement = E^\complement \cup F^\complement $$

\subsection{Räkneregler}
Om både mamman och pappan är av genotypen (blå, brun) (d.v.s. har ett blått och ett brunt anlag), så är avkommans gensuppsättning en av följande fyra:\\
(blå, blå), (blå, brun), (brun, blå), (brun, brun), vilket också är utfallsrummet.\\
Intressant skulle vara att studera händelsen B=\{ (blå, blå), (blå, brun), (brun, blå) \} (minst ett blå anlag).
och händelsen A=\{ (blå, brun), (brun, blå), (brun, brun) \} (minst ett brun anlag).\\
Beskriv (med ord):
\begin{itemize}
    \item $A\cap B$ "precis ett blå och ett brunt anlag"
    \item $A\cup B$ "hela utfallsrummet"
\end{itemize}

\end{document}
